\chapter{Dual Spaces}
Consider a vector space $V$ over a field $\mathbb{F}$ with basis $\{\vec{e}_\mu\}$.
The \emph{Dual Space} of $V$, denoted $V^*$, is the set of all linear maps from $V$ to $\mathbb{F}$:
\begin{align*}
    V^* = \{\phi : \phi(\vec{v}) \in \mathbb{F}, \phi \text{ is linear}\}.
\end{align*}
Elements of $V^*$ are known as \emph{covectors}. Note that if we've created $V$, we don't need to do anything to create $V^*$; it is automatically created by $V$.

This dual space is itself a vector space, since we can give it a concept of addition and scalar multiplication, and the zero covector.
\begin{itemize}
    \item Addition: Consider covectors $\Lambda$ and $Z$ of the dual space of $V$ (so they are linear maps from $V$ to $\mathbb{F}$), and a vector $\vec{v} \in V$. We can say that $\langle \Lambda + Z, \vec{v} \rangle = \langle \Lambda,\vec{v} \rangle + \langle Z, \vec{v} \rangle$.
    \item Scalar multiplication: Again with a covector (linear map) $\Lambda$, we can say that $\langle a\Lambda, \vec{v} \rangle = a\langle \Lambda, \vec{v} \rangle$, where $a$ is a scalar from $\mathbb{F}$.
    \item Zero vector: Consider the map which sends every vector to $0_\mathbb{F}$; this can act as a `zero covector'.
\end{itemize}

\subsection{Dual Basis}
When we create a vector space $V$ with basis $\{\vec{e}_\mu\}$, we automatically create a dual space $V^*$; since the dual space is also a vector space, it must have a basis. We call this basis $\{\vec{e}^\mu\}$, similar to the basis for the original vector space, but with a superscript index instead of a subscript.
Like our choice for the basis of $V$, the choice of $\{\vec{e}^\mu\}$ is arbitrary.
But, while there isn't a specific basis created, we can choose our bases selectively.
Remember that $\vec{e}^\mu$ are \emph{linear maps} which can act upon $\vec{e}_\mu$;
We want to choose the basis for our dual space such that
\begin{align*}
    \langle \vec{e}^0, \vec{e}_0 \rangle &= 1, \\
    \langle \vec{e}^1, \vec{e}_1 \rangle &= 1, \\
    &\vdots \\
    \langle \vec{e}^\mu, \vec{e}_\mu \rangle &= 1,
\end{align*}
but 
\begin{align*}
    \langle \vec{e}^0, \vec{e}_1 \rangle &= 0, \\
    &\vdots \\
    \langle \vec{e}^\mu, \vec{e}_\nu \rangle &= 0.
\end{align*}
We can generalize this statement to
\begin{align*}
    \langle \vec{e}^\mu, \vec{e}_\nu \rangle = \delta^\mu_\nu,
\end{align*}
where $\delta^\mu_\nu$ is the Kronecker delta function,
\[
    \delta^\mu_\nu =
    \begin{cases}
        1 & \mu = \nu \\
        0 & \mu \not= \nu
    \end{cases}.
\]
This choice is also rather arbitrary --- we just like dealing with the case when $\langle \vec{e}^\mu, \vec{e}_\nu \rangle = 1$ if and only if $\mu = \nu$, and is zero otherwise. It's also important to realize that exactly \emph{which} linear maps $\{\vec{e}^\mu\}$ are is not specified; although the creation of $V$ implies the creation of $V^*$, there isn't (yet) a way to relate vectors in $V$ to vectors in $V^*$.

\subsection{The Dual Dual Space}
Since the dual space $V^*$ is a vector space in its own right, it too must have a dual space, the set of all linear maps which send a vector in $V^*$ to $\mathbb{F}$.
If we look at the bracket notation for what we're really doing, it becomes obvious what this space must be:
\begin{align*}
    \langle \Lambda, \cdot \rangle &: V \to \mathbb{F} \\
    \langle \cdot, A^\mu \vec{e}_\mu \rangle &: V^* \to \mathbb{F}.
\end{align*}
If \emph{covectors} go from $V$ to $\mathbb{F}$, then \emph{vectors} go from $V^*$ to $\mathbb{F}$, so $V^{**} = V$.
Just like how we can build any vector in $V$ using Einstein summation over the basis of $V$, we can build a basis for $V^*$ in the same way. Since the basis vectors for $V^*$ are of the form $\{\vec{e}^\mu\}$, we write an arbitrary vector in the dual space as
\[ \Lambda = A_\mu \vec{e}^\mu \in V^*. \]
This makes the above definition for linear maps
\begin{align*}
    \langle A_\mu \vec{e}^\mu, \cdot \rangle &: V \to \mathbb{F}, \\
    \langle \cdot, A^\mu \vec{e}_\mu \rangle &: V^* \to \mathbb{F}.
\end{align*}

\subsection{Covariance and Contravariance}
If we start with a vector space $V$ and its corresponding dual space $V^*$ (again, which is which is a rather arbitrary choice, sense \emph{both} are vector spaces), we say that elements of the vector space $V$ are \emph{vectors}, and they are said to vary \emph{contravariantly}, while elements of the dual space $V^*$ are \emph{covectors} which vary \emph{contravariantly}.

\subsection{Arbitrary Vectors and Mappings}
Imagine we have an arbitrary covector $B^\mu \vec{e}_\mu$, and an arbitrary vector $A_\nu \vec{e}^\nu$ upon which this covector acts. We write this as 
\[ \langle B_\mu \vec{e}^\mu, A^\nu \vec{e}_\nu \rangle. \]
Since these mappings are linear, we can factor out the coefficients, leaving
\[ B_\mu A^\nu \langle \vec{e}^\mu, \vec{e}_\nu \rangle. \]
Since we know that $\langle \vec{e}^\mu, \vec{e}_\nu \rangle = \delta^\mu_\nu$, we can simplify this into
\[ B_\mu A^\nu \langle \vec{e}^\mu, \vec{e}_\nu \rangle = B_\mu A^\nu \delta^\mu_\nu = B_\mu A^\mu. \]
This means that we can refer to the result of the covector-vector applications by just the coefficients, as long as we define the dual basis by the delta function.
Note also that $B_\mu A^\mu$ is a member of the field $\mathbb{F}$ over which $V$ and $V^*$ are taken.