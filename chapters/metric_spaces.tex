\chapter{Metric Spaces}
Up until this point, everything we've discussed has risen out of just the original vector space $V$; we created the dual space, vectors, covectors, tensor product spaces, and basis transformations based only on the original definition of $V$, its scalar field $\mathbb{F}$, and its basis $\vec{e}_\mu$.
Now, we're going to add a new object into the mix.
This will be the \emph{inner product} or \emph{metric}, denoted $\rho: V \times V \to \mathbb{F}$.
Vector spaces which are equipped with this function are known as \emph{metric spaces} or \emph{inner product spaces}. 
We want this function to satisfy some key properties.
\begin{enumerate}
    \item $\rho$ must map an ordered pair of vectors to the scalar field of $V$.
    \item $\rho$ must be linear in both arguments; then $\rho(\lambda\vec{v},\vec{u}) = \lambda \rho(\vec{v},\vec{u})$ and $\rho(\vec{v} + \vec{u},\vec{w}) = \rho(\vec{v},\vec{w}) + \rho(\vec{u},\vec{w})$.
    \item[3*.] $\rho$ must be symmetric; $\rho(\vec{v},\vec
    u) = \rho(\vec{u},\vec{v})$.
    \item[4*.] $\rho(\vec{v},\vec{v}) \geq 0$, and $\rho(\vec{v},\vec{v}) = 0$ if and only if $\vec{v} = \vec{0}$. This is known as a \emph{positive-definite} form. If this rule does not hold, it is a \emph{indefinite} form.
    If $\rho$ is positive-definite, then $V$ is a \emph{Reimannian} metric space; if it is indefinite, $V$ is a \emph{pseudo-Reimannian} metric space.
\end{enumerate}
These last two conditions are marked with asterisks because the definition of $\rho$ when used in general relativity differs from the definition of $\rho$ when discussing pure mathematics.
In general relativity, we insist that $\rho$ be symmetric, but we don't require that $\rho(\vec{v},\vec{v})$ be greater than or equal to 0, so General Relativity is based on a pseudo-Remannian metric space.
This function is a specific function, not a set of all possible functions like the dual space is.
By adding this specific function, we promote $V$ to become a metric space.

There are a few notational conventions for an inner product.
\begin{itemize}
    \item A named function, like $\rho(\cdot,\cdot)$.
    \item Using ordered pairs, like $(\cdot, \cdot)$.
    \item The Bra-Ket notation of quantum mechanics, $\langle \cdot \mid \cdot \rangle$.
    \item The braket notation which we have so far used to mean mappings, $\langle \cdot, \cdot \rangle$.
    \item The elementary vector notation of the dot product, where $\vec{v}\cdot\vec{u}$ is the inner product of $\vec{v}$ and $\vec{u}$.
\end{itemize}
Throughout this lesson, we will use the ordered pair notation of an inner product.

\subsection{Creating the Inner Product}
We define the inner product by what defining what happens to the basis vectors of $V$.
Since the metric is linear, we can simply factor out the scalars and use these definitions to calculate the inner product of any two vectors.
In special relativity, we use the \emph{Minkowski metric}\footnote{Some books take the reverse convention, where $\eta_{00} = 1$ and $\eta_{11}, \eta_{22}, \eta_{33} = -1$. This choice between $(-,+,+,+)$ and $(+,-,-,-)$ is known as the \emph{metric signature}.} $\eta$, where 
\[ (\vec{e}_\mu,\vec{e}_\nu) = \eta_{\mu\nu},\quad \eta_{\mu\nu} = 
    \begin{cases}
        -1 & \mu = \nu = 0 \\
        1 & \mu = \nu \in \{1,2,3\} \\
        0 & \mu \not= \nu
    \end{cases}.
\]
This can be represented by a $4\times4$ matrix with each element having the value of $\eta_{r,c}$, like
\[
	\eta = 
    \begin{bmatrix}
        -1 & 0 & 0 & 0 \\
         0 & 1 & 0 & 0 \\
         0 & 0 & 1 & 0 \\
         0 & 0 & 0 & 1
    \end{bmatrix}.
\]
In general relativity, however, we don't have the luxury of picking such nice values.
Instead, we look to what values nature seems to prefer.
