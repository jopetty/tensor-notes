\chapter{The Cartesian Product}
The \emph{Cartesian product} $\times$ is a binary operation on sets which produces a new set containing all ordered pairs where the first element comes from the left-hand set and the second element comes from the right-hand set. If $A = \{a,b,c,\dots\}$ and $\Gamma = \{\alpha,\beta,\gamma,\dots\}$, then $A \times \Gamma = \{(a,\alpha), (a,\beta), \dots, (b,\alpha), \dots\}$.
Specifically, the Cartesian product is defined as
\[ A \times B = \{(a,b) : a \in A, b \in B\}. \]
While we call it a binary operation, we can have an arbitrary number of sets upon which it can act; consider the generalization to $n$ sets
\[ A \times B \times C \times \cdots = \{(a,b,c,\dots) : a \in A, b \in B, c \in C, \dots\}, \]
which produces an $n$-tuple. 
We use the Cartesian product to describe the domain and range of new binary functions that we create. If we imagine a new binary operation $\odot$ which acts upon sets $A$ and $B$ and maps to $C$, we could define it as
\[ \odot : A \times B \to C. \]
For the same of brevity, we often use an exponent to represent the number of times a Cartesian product has been taken over the same vector space, so
\[ A \times A \times A = A^3 \]
and
\[A  \times A \times B \times B \times B = A^2 \times B^3. \]
